%% LyX 2.1.4 created this file.  For more info, see http://www.lyx.org/.
%% Do not edit unless you really know what you are doing.
\documentclass[10pt,a4paper]{report}
\usepackage[latin1]{inputenc}
\setcounter{secnumdepth}{3}
\setcounter{tocdepth}{3}
\usepackage{amsmath}
\usepackage{amssymb}

\makeatletter

%%%%%%%%%%%%%%%%%%%%%%%%%%%%%% LyX specific LaTeX commands.
\pdfpageheight\paperheight
\pdfpagewidth\paperwidth


%%%%%%%%%%%%%%%%%%%%%%%%%%%%%% User specified LaTeX commands.
\usepackage{amsfonts}
\usepackage{graphicx}
\usepackage{titlesec}
\usepackage{blindtext}\usepackage{color}\definecolor{gray75}{gray}{0.75}
\newcommand{\hsp}{\hspace{18pt}}
\titleformat{\chapter}[hang]{\Huge\bfseries}{\thechapter\hsp\textcolor{gray75}{|}\hsp}{0pt}{\Huge\bfseries}
\author{Losio Davide Francesco, Luchetti Mauro, Mosca Paolo}
\title{myTaxiService\\Software Design Document\\ Version 1.0}

\makeatother

\begin{document}
\maketitle \tableofcontents{}


\chapter{Introduction}

This documentation will be used to aid in software development by
providing further details of how the software should be built. Within
the Software Design Document are narrative and graphical documentation
of the software design of the project including use case models, sequence
diagrams, and other supporting requirement information.


\section{Purpose}

The purpose of the Software Design Document is to provide a description
of MyTaxiService system design and architecture fully enough to allow
software development to proceed with an understanding of what is to
be built and how it is expected to be built. To achieve this DD(\textbf{D}esign
\textbf{D}ocument) translates and states more accurately the Requirement
Specifications described in the MyTaxyService RASD document. It identifies
high-level system architecture and design framework as well as hardware,
software, communication and interface components.


\section{Scope}

MyTaxiService application is a server/client combination that will
allow a user to handle different type of taxi service, keeping track
of all the transaction necessary for the completion of each operations.
This will include booking a taxi, request a taxi as soon as possible,
the handling of the sharing option and, for the taxi drivers, the
managing of the taxi queue. All this functionalities will be guaranteed
in the way and in the manner explained in the RASD document. Via a
Cross Platform Web Environment (by the use of angularJS, Ionic, Cordova
and nodeJS frameworks), the MyTaxiService will be able to run on various
platforms, including Unix, Linux and Windows based systems, and all
the portable devices based on Android and Ios. When a network connection
to the server is available, the user will be able to synchronize his
PD (\textbf{P}ortable \textbf{D}evice) or PC with the server, he will
be able to log in or register and makes his own request in the case
of the passenger-user. Or to set is availability, accepts or rejects
request in the case of the taxi-driver user.\\
\\
 Below are stated some main issues with which the system has to be
capable to cope with.
\begin{itemize}
\item \textbf{PD Issues:} Because of memory limitations, a PD will only
store data and application parts that are strictly necessary for a
PD user. Also, PDs have reduced screen size and limited input capability
compared to PCs, so we will design PD standalone functionality in
manner that can be easily presented on a typical 240x320 screen.
\item \textbf{Synchronization:} We will implement server software to serve
as an interface between the PC or PD and the Application logic, by
the re-using of already existent services offered by third parties
company.
\item \textbf{Transaction and functionalities:} all the transactions and
booking procedure will be handled by the application logic layer that
will be divided from the presentation layer, as well as the queue
managing features and algorithm. PDs will implement only presentation
layer and connection functionalities. 
\end{itemize}
The architecture will be developed and structured to support the fulfillment
of this main issues. 


\section{Definitions, Acronyms, Abbreviations}
\begin{itemize}
\item {\textbf{RASD:}} \textbf{R}equirements \textbf{S}pecification \textbf{A}nalysis
\textbf{D}ocument 
\item {\textbf{DD:}} \textbf{D}esign \textbf{D}ocument 
\item {\textbf{PD:}} \textbf{P}ortable \textbf{D}evice 
\item {\textbf{MVC:}} \textbf{M}odel \textbf{C}ontrol \textbf{V}iew
\end{itemize}

\section{References}
\begin{itemize}
\item {\textbf{MyTaxyService RASD}} - November/6/2015, edited by Dadoz+Grin-Go+Pol
Corporation; 
\end{itemize}

\section{Document Structure and Overview}
\begin{itemize}
\item {\textbf{Architectural Design:}} this section is the main focus
of this document. It provides an overview of the system's major components
and architecture, as well as architectural styles, pattern used and
other design decision.\\



a detailed analysis of modules will also describe lower-level classes,
components, and functions, as well as the interaction between these
internal components.\\


\item {\textbf{Algorithm design:}} this section is focused on the definition
of the most relevant algorithm of the project.
\item {\textbf{User interface design:}} this section provide an overview
on how the user interface(s) of the system will look like. In particular,
referring on what has already stated in the RASD, here some further
details is specified.
\item {\textbf{Requirements Traceability:}} this section explain how the
requirements defined in the RASD map into the design elements that
are defined in this document. 
\end{itemize}

\chapter{Architectural Design}


\section{Overview}

This sections analyze several different part of the design architecture: 
\begin{itemize}
\item {System Components And Their Interactions:} here are individuated
and stated the main system software components in which the software
is split into, and in the interactions among them. To achieve this
latter feature, will be provided a high level description of the interfaces
to be set-up. Within this description are: \subitem Component; \subitem
Deployment; \subitem Runtime;\\
\\
 UML diagrams. These diagrams are intended as a supply to the better
understanding, as well as a more clear and simply specification of
our components division. 
\item {Architectural Styles And Patterns} here are listed the architectural
styles and patterns to use to solve the main interactions and functionality
problems. They are taken in this first version of the DD more as suggestions
and ideas. They will be updated and revised during the developing
of the application where requested by the circumstances. In the sense
that if the patterns and architectural styles will actually reveal
to be unfeasible, different approach will be evaluated and chosen. 
\item {Other Design Decision:} Some other decision that doesn't belong
to other section are listed here. 
\end{itemize}

\section{System Components }

The system to be produced is a client/server system with the common
MVC pattern. The components are split in order to represent this logic.
For each component the main modules are listed.


\subsection{Server application logic}

This component represent the core logic of the whole system. Modules
here contained are responsible for the correct features functioning
of MyTaxiService. 


\subsubsection{Queue Handler}

This module has to guarantee the right usage of the queues into the
application. The operations that this module handles are: add a new
driver to the queue, delete a driver from the queue, select the first
driver in the queue and move a driver from top to the bottom of the
queue.


\subsubsection{Requests Handler}

This module is responsable of the correct forwarding of each request
coming from a user. It is in straight contact with the Queue Handler
module because of his necessity to send requests to drivers.


\subsubsection{Notification Center}

This module manages any notification generated by a user, wheather
it is a passenger or a taxi driver, or by the system itself, as in
the case of error messages. The operations that this module handles
are: generate an error messa. \\\\


\subsection{Server communicator}

This component represent the part of the system that is directly in
contact with the users.


\subsubsection{User communicator}

This module is responsible of the correct forwarding information between
the server and a user. The operations that this module handles are:
receive a message from a user, send a message to a user, handle timout
errors.


\subsubsection{Driver communicator}

This module is responsible of the correct forwarding information between
the server and a driver. The operations that this module handles are:
receive a message from a driver, send a message to a driver, handle
timout errors.
\end{document}
